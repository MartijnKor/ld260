\documentclass{article}
\title{Cheatsheet Linear Algebra 2LD60}
\author{Martijn Kortenhoeven}

\usepackage[margin=0.5in]{geometry}
\usepackage{tabularx}
\usepackage{amsmath}
\usepackage{amssymb}
\usepackage{bm}
\renewcommand{\arraystretch}{1.2}

\makeatletter
\renewcommand*\env@matrix[1][*\c@MaxMatrixCols c]{%
   \hskip -\arraycolsep
   \let\@ifnextchar\new@ifnextchar
   \array{#1}}
\makeatother


\begin{document}
\maketitle

\section{Chapter one}
\section{Chapter two}
\section{Chapter three}


\begin{tabularx}{\textwidth}{p{40mm}Xp{55mm}}
\textbf{Vector space} & A set of vectors for which the set of axioms and closure condition on page 131 are satisfied. & C1: $\textbf{x} + \textbf{y} \in V$ \newline C2: $\alpha \textbf{x} \in V$ \\

\hline
\textbf{Subspace} & A subset of a vector for which C1 and C2 also hold, but then for the subset of that vector. & C1: $\textbf{x} + \textbf{y} \in S$ \newline C2: $\alpha \textbf{x} \in S$ \newline See example \ref{subspace} \\

\hline
\textbf{Null space matrix} & Solve the matrix for a solution equal to zero. & See example \ref{nullspace} \\

\hline
\textbf{Commuting matrices} & Matrices A and B are said to be commuting if AB = BA. \\

\hline
\textbf{Spanning set for a \newline vector space} & The set $\{\textbf{v}_1, ..., \textbf{v}_n\}$ is a spanning set for V if and only if every vector in V can be written as a linear combination of $\textbf{v}_1,..., \textbf{v}_n$.

\end{tabularx}

\section{Examples}
\subsection{Subspace} \label{subspace}
Given is the following set of vectors: $S = \left\{ \begin{bmatrix}
x \\ 1
\end{bmatrix} | \ x \text{ is a real number} \right\}$. To determine whether it is a subspace of $\mathbb{R}^2$ it has to fulfill closure conditions C1 and C2. Additionally it has to fulfill the property where x is a real number and the second row is equal to 1.
\begin{equation*}
\begin{bmatrix}
x \\ 1
\end{bmatrix}
+
\begin{bmatrix}
y \\ 1
\end{bmatrix}
\not\in S \ \text{when } \alpha \neq 1
\end{equation*}
Therefore it is not a subspace. But both conditions fail to hold.
\begin{equation*}
\alpha \begin{bmatrix}
x \\ 1
\end{bmatrix}
=
\begin{bmatrix}
\alpha x \\ \alpha
\end{bmatrix}
\not\in S
\end{equation*}

\subsection{Null space} \label{nullspace}
Determine N(A) if
\begin{equation*}
A = \begin{bmatrix}
1 & 3& -4 \\
2 & -1 & -1 \\
-1 & -3 & 4
\end{bmatrix}
\end{equation*}
Use reduction techniques to solve for $A\textbf{x}= \textbf{0}$.
\begin{equation*}
A \Rightarrow \begin{bmatrix}[ccc|c]
1 & 3 & -4 & 0\\
0 & -7 & 7 & 0 \\
0 & 0 & 0 & 0
\end{bmatrix}
\end{equation*}
So, $x_1 = x_2 = x_3$. Therefore the vector space N(A) consists of all vectors of the form
\begin{equation*}
\alpha \begin{bmatrix}
1 \\ 1 \\ 1
\end{bmatrix}
\end{equation*}

\end{document}


